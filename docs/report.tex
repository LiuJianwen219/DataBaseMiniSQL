\documentclass[UTF8]{ctexart}
\usepackage{graphicx}
\usepackage{geometry}
\usepackage[section]{placeins}
\usepackage{hyperref}
\usepackage{ulem}
\hypersetup{
colorlinks=true,
linkcolor=black
}
\geometry{a4paper,scale=0.8}


\title{\huge{DB-MiniSQL}}
\author{姓名}
\date{2020/05/22}
\begin{document}
\maketitle

\tableofcontents

\clearpage
\section{前言}



\clearpage
\section{需求分析}
\subsection{概述}
\begin{enumerate}
    \item \textbf{数据类型}: int, char(n), float, 其中$1\le n\le255$。
    \item \textbf{表定义}: 一个表最多定义\textit{32}个属性,可以指定是否为\textit{unique},支持\textit{主键定义}。
    \item \textbf{索引的建立和删除}: 对于表的主属性自动建立B+树索引,对于声明为unique的属性可以通过SQL语句由用户指定建立/删除B+树索引。
    \item \textbf{查找记录}: 可以通过指定用\textit{and}连接的多个条件进行查询,支持\textit{等值查询和区间查询}。
    \item \textbf{插入和删除记录}: 支持每次一条记录的插入操作;支持每次一条或多条记录的删除操作。
\end{enumerate}




\clearpage
\section{模块设计}
\subsection{Interpreter $\&$ API}



\end{document}
